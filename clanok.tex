
\documentclass[10pt,twoside,slovak,a4paper]{article}

\usepackage[slovak]{babel}
%\usepackage[T1]{fontenc}
\usepackage[IL2]{fontenc} 
\usepackage[utf8]{inputenc}
\usepackage{graphicx}
\usepackage{url} 
\usepackage{hyperref} 

\usepackage{cite}


\pagestyle{headings}

\title{Umelá inteligencia v autonómnych autách a iných vozidlách\thanks{Semestrálny projekt v predmete Metódy inžinierskej práce, ak. rok 2021/22, vedenie: Zuzana Špitálová}} 

\author{Jakub grúber\\[2pt]
	{\small Slovenská technická univerzita v Bratislave}\\
	{\small Fakulta informatiky a informačných technológií}\\
	{\small \texttt{xgruber@stuba.sk}}
	}

\date{\small 12. október 2021} 


\begin{document}

\maketitle

\begin{abstract}
\ldots
\end{abstract}
Umelá inteligencia je veľmi dôležitou súčasťou moderného výskumu. Začína sa využívať v mnohých oblastiach a jednou z týchto oblastí je aj automobilový priemysel. Cieľom je vytvoriť autonómne vozidlo, ktoré bude schopné jazdiť s minimálnymi alebo žiadnymi zásahmi šoféra. V mojej práci by som sa chcel zamerať na to, ako funguje umelá inteligencia v autonómnych autách (aké úlohy vykonáva, čo spracúva...), aké postupy a prostriedky sa na jej fungovanie využívajú (neurónové siete, veľké dátové úložiská... ) , aký je stav v oblasti implementácie umelej inteligencie do autonómnych áut, ako sa auto „učí“ samo šoférovať (proces zdokonaľovania reakcii umelej inteligencie auta na rôzne situácie), morálne otázky, ktoré z implementácie umelej inteligencie do áut vyvstávajú, aké problémy sa pri jej implementácii vyskytujú a dopady využívania autonómnych áut na spoločnosť. \cite{PLP-Framework}


\section{Úvod} \label{uvod}
V mojej práci sa zameriavam na umelú inteligenciu v autonómnych autách. Je to oblasť, ktorá je pomerne nová, no boli v nej učinené veľké pokroky. Autonómne autá sa pomaly ale isto stávajú súčasťou bežného života. Dokážu vykonávať celé množstvo úkonov bez zásahu vodiča a vo svete je snaha ich čoraz viac implementovať do rôznych oblastí, či už ide o každodennú dopravu, hromadnú dopravu, upravené vozidlá pre zdravotne znevýhodnených a rôzne iné vozidlá pre rozmanité špecifické úkony \cite{Fieldsofapplication} . Moja práca bude rozdelená na časti, kde sa budem venovať týmto otázkam a problematikám: \begin{enumerate} \item Ako funguje umelá inteligencia v autonómnych vozidlách  (klasifikácia autonómnych vozidiel, aké ovládacie prvky a technológie vozidlo používa, funkcia umelej inteligencie a ako sa vozidlo "učí" samo šoférovať)
\item Stav implementácie umelej inteligencie do automobilového sektora (problémy pri implementácii umelej inteligencie do vozidiel)
\item Dopady na spoločnosť (morálne otázky vyplývajúce z implementácie umelej inteligencie do vozidiel)
\end{enumerate}

\section{Ako funguje umelá inteligencia v autonómnych vozidlách}

\subsection{Klasifikácia autonómnych vozidiel}
Autonómne vozidlá sú roztriedené do šiestich rôznych úrovní podľa klasifikácie J3016 zverejnenej v roku 2014 (posledná aktualizácia 1.7.2019) asociáciou SAE (Society of Automotive Engineers). Vozidlá sú podľa klasifikácie rozdelené do nasledovných úrovní: \begin{itemize}
\item Úroveň 0 (žiadna automatizácia): do tejto úrovne spadá väčšina dnešných vozidiel. Všetky úkony sú vykonávané vodičom. Sú to napríklad točenie volantom, akcelerácia, brzdenie, monitorovanie okolia, navigácia, odbáčanie a dávanie signálu o odbáčaní smerovkou, zmena jazdného pruhu. Možu byť prítomné niektoré varovné systémy ako kontrola slepého bodu alebo upozornenie pred kolíziou. 
\item Úroveň 1 (asistencia vodiča): vozidlo môže točiť volantom, zrýchľovať aj brzdiť ale nie vo všetkých situáciach. Vodič musí byť vždy pripravený po výzve vozidla zasiahnuť. To znamená, že musí sledovať dopravnú situáciu a byť ostražitý.
\item Úroveň 2 (čiastočná asistencia vodiča): vozidlo môže točiť volantom, zrýchľovať aj brzdiť (vykonáva ovládanie), ale nechá vodiča prevziať kontrolu ak si vodič všimne objekt alebo situáciu na ktorú vozidlo nereaguje. V týchto prvých troch úrovniach je vodič povinný sledovať dopravnú situáciu a podmienky okolo seba.
\item Úroveň 3 (podmienečná asistencia): vozidlo monitoruje okolie a vykonáva ovládanie ale len v určitých prostrediach a situáciach, ako sú napríklad diaľnice. Vodič však musí byť pripravený zasiahnuť ak ho k tomu vozidlo vyzve.
\item Úroveň 4 (výsoká automatizácia): vozidlo monitoruje okolie a vykonáva ovládanie vo viacerých prostrediach a situáciach, nie však pri všetkých, napríklad pri zlých poveternostných podmienkach. Pri bezpečnej situácii šofér zapne automatické ovládanie, potom už nie je jeho zásah pri chode vozidla potrebný.
\item Úroveň 5 (plná automatizácia): vodič zadá len destináciu a naštartuje vozidlo. Vozidlo sa stará o všetky ostatné úlohy. Vozidlo sa presunie do zadanej destinácie, ak táto destinácia nemá obmedzenia a všetky úkony a rozhodnutia robí samo. (Standard SAE J3016, 2019) \cite{Howautonomouscarswork}
\end{itemize}

\subsection{Ovládacie prvky a technológie autonómneho vozidla}
Autonómne vozidlá pri svojej prevádzke používajú okrem umelej inteligencie množstvo ďaľších technológii a ovládacích prvkov. Vstupy (dáta) z týchto technológii spracúva počítač vysokou rýclosťou a na základe získaných informácii za pomoci softvéru pošle výstupné dáta do príslušných elektromechanických jednotiek, ktoré majú na starosť automatické točenie volantom, brzdenie a akceleráciu. Tento počítač je umiestnený vo vozidle a je napojený na GPS a internet aby mal dostupné relevantné dáta v reálnom čase. Technológie, ktoré zabezpečujú zisk dát spracovávaných počítačom sú nasledovné:
GPS (Global Positioning system): je satelitný navigačný systém, ktorý poskytuje informácie o polohe a čase kdekoľvek na zemi, kde nič nebráni prenosu zo satelitov do GPS zariadenia. Polohu naraz určujú najmenej štyri satelity. GPS zabezpečuje koordináciu vozidla na trase, informácie o jeho polohe. GPS doplňuje IMU (Inertial measurement unit). Je to elektronické zariadenie, ktoré používa kombináciu akcelometrov, gyroskopov a magnetometrov aby napomáhal GPS keď nie je dostupný signál napríklad v tuneloch, za zlého počasia alebo keď je prítomná elektromagnetická interferencia.\cite{Howautonomouscarswork}

Ultrasonické senzory: sú namontované po stranách vozidla. Merajú pozíciu a vzdialenosť objektov pri parkovaní za pomoci ultrazvukových vĺn, varujú pred kolíziou a prechodom cez pruhy.\cite{Howautonomouscarswork}

Videokamery: sú namontované na prednom skle a poskytujú trojdimenzionálny obraz v reálnom čase. Rozpoznávajú dopravné značenie, chodcov, zvieratá atď.\cite{Howautonomouscarswork}

LIDAR (Light detection and Ranging): je zariadenie umiestnené na streche vozidla, ktoré detekuje a sleduje objekty a vytvorí 3D mapu okolia. Meria vzdialenosť svetelným lúčom a analyzuje odrazené svetlo. Dáta následne posiela do počítača na spracovanie a vytvorenie mapy. Skladá sa z vysielača laserového lúča, zrkadla a prijímača laserového lúča. Pokrýva dĺžku 32 metrov a má 360-stupňový rozsah.\cite{Howautonomouscarswork}




\bibliography{literatura}
\bibliographystyle{plain} 
\end{document}
